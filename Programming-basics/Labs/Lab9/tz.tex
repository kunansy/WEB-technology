\documentclass[a4paper,12pt]{report}

\usepackage{cmap} % поиск в документе
\usepackage[T2A]{fontenc} % кодировка
\usepackage[utf8]{inputenc} % кодировка исходного текста
\usepackage[english, russian]{babel} % локализация и переносы

\usepackage{indentfirst} % отступ после заголовка
\usepackage{hyperref}

\author{Колобов Кирилл, Сметанина Александра, 201-321}
\title{ТЗ к проекту <<Напоминалка: возьми зонтик>>}
\date{25 декабря 2020 г.}

\renewcommand{\thesection}{\arabic{section}}

\begin{document}
\maketitle

\tableofcontents
\newpage

\section{Описание}
Проект представляет из себя мобильное приложение, которое напоминает пользователю 
взять зонт в случае прогноза на дождливую погоду.

\section{Функциональность}
\begin{enumerate}
\item{Работает на основании прогноза погоды и текущей погоды, полученных из API Яндекс.Погоды\footnote{
      Другому API, если получение ключа для Яндекс.Погоды окажется невозможным}.}
\item{При запуске показывает три секции (утро, день, вечер) формата: 
      сообщение (необходим ли зонт) -- прогноз (ливень, град, солнце, снег etc и температура).
      Но фоне каждой секции при этом при стоит картинка, соответствующая 
      прогнозу: картинка облаков для <<облачно>>, картика с дождём для <<дождь>> etc.

      При этом секции тех частей дня, которые на момент запуска 
      уже прошли, преобразованы в ч/б формат.}
\item{Обращается к API Яндекс.Погоды, 
      получает оттуда прогноз на неделю и сохраняет его в файл формата JSON.}
\item{Если приложение запускается, но интернет-соединения нет, 
      то сообщение формируется на основе сохранённого JSON.}
\item{Если приложение запускается, и интернет есть, то происходит 
      обновление файла JSON, сообщение формируется на основе полученного ответа.}
\end{enumerate}

\section{Требования}
\begin{enumerate}
\item{Код должен быть написан с соблюдением стандарта кодирования \textbf{PEP8}.}
\item{К коду должны быть написаны модульные тесты.}
\item{В проекте должно быть предусмотрено логгирование в файл различных 
    сообщений с указанием их уровня, даты и времени вызова, имени функции и
    модуля, в которых вызван логгер, id текущего процесса.}
\item{Ключ для доступа к API должен храниться в config файле в директории приложения.}
\item{Код должен быть корректно структурирован на функции/классы/модули, 
    задокументирован для дальнейшего масштабирования.}
\item{Должна быть выполнена обработка ошибок:}
\begin{itemize}
\item{Отсутствие интернет-соединения при запуске приложения.}
\item{Ошибки при получении ответа от API.}
\item{Отсутствие config файла c ключом к API.}
\end{itemize}
\end{enumerate}
\end{document}
