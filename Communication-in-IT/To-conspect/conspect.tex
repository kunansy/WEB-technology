\documentclass[a4paper,12pt]{report}

\usepackage{cmap} % поиск в документе
\usepackage[T2A]{fontenc} % кодировка
\usepackage[utf8]{inputenc} % кодировка исходного текста
\usepackage[english, russian]{babel} % локализация и переносы

\usepackage{hyperref} % ссылки -> гиперссылки

\author{Колобов Кирилл}
\title{Глава 11}
\date{11 November 2020}

\begin{document}
\setcounter{chapter}{10}
\chapter{Кофейные тайны. Оформление и форматирование текста}
\section{Введение}
В этой главе автор описывает способы форматирования текста, облегчающие его восприятие читателем: 
\begin{itemize}
    \item{Воздушные абзацы, \ref{air-par:main}}
    \item{Буллеты, \ref{bullet}}
\end{itemize}

\section{Воздушные абзацы}\label{air-par:main}
Разделённый на абзацы текст выглядит более объёмным из-за отступов и увеличенных межстрочных интервалов, но менее <<загруженным>>, что облегчает его чтение. 

Вызвано это тем, утверждает автор, что расчленённый на абзацы текст помогает читателю зрительно оценить, сколько контента его ждёт. 

\section{Буллеты}\label{bullet}
\textbf{Буллет} -- так автор называет маркированные списки. 
Она рекомендует содержимое текста структурировать с их помощью, где это возможно.
Если пунктов всего 3-5, то стоит использовать ненумерованные списки, 
если их больше -- нумерованные.

\tableofcontents

\end{document}
