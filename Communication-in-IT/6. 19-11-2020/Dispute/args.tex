\documentclass[a4paper,12pt]{report}

\usepackage{cmap} % поиск в документе
\usepackage[T2A]{fontenc} % кодировка
\usepackage[utf8]{inputenc} % кодировка исходного текста
\usepackage[english, russian]{babel} % локализация и переносы

\usepackage{hyperref}

%%% Работа с картинками
\usepackage{graphicx}  % Для вставки рисунков
%\setlength\fboxsep{3pt} % Отступ рамки \fbox{} от рисунка
%\setlength\fboxrule{1pt} % Толщина линий рамки \fbox{}
%\usepackage{wrapfig} % Обтекание рисунков и таблиц текстом

\author{Колобов Кирилл, 201-321}
\date{27 November 2020}
\title{Аборты: геноцид или выбор?}

\begin{document}

\maketitle

\tableofcontents

\chapter{Введение}

\chapter{За аборты}
\section{Зигота -- не человек}
\section{Человек начинается с рождения}
\section{Зигота -- часть тела матери}
\section{У зиготы нет разума}
\section{У матери есть выбор}
\section{Ты хочешь плодить нищету?}
\section{Мать не хотела заводить ребёнка}
\section{Нет дня зачатия, есть день рождения}
\section{}
\section{}
\section{}

\chapter{Против абортов}
\section{Зигота -- человек}
\section{Человек начинается с зачатия}
\section{Зигота -- новый организм}
\section{У тебя тоже}
\section{Можно убить новорожденного?}
\section{А чем думала мать?}
\section{А чем думала мать?}
\section{Контингентно культуре}
\section{}
\section{}
\section{}

\chapter{Выводы}

\end{document}

