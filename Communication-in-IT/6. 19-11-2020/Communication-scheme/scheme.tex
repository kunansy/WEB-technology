\documentclass[a4paper,12pt]{report}

\usepackage{cmap} % поиск в документе
\usepackage[T2A]{fontenc} % кодировка
\usepackage[utf8]{inputenc} % кодировка исходного текста
\usepackage[english, russian]{babel} % локализация и переносы

\usepackage{hyperref}

%%% Работа с картинками
\usepackage{graphicx}  % Для вставки рисунков
%\setlength\fboxsep{3pt} % Отступ рамки \fbox{} от рисунка
%\setlength\fboxrule{1pt} % Толщина линий рамки \fbox{}
\usepackage{wrapfig} % Обтекание рисунков и таблиц текстом

\author{Колобов Кирилл, 201-321}
\title{Схема коммуникации}
\date{}

\begin{document}
\maketitle

\chapter*{Схема}
\includegraphics[scale=0.1]{scheme.jpg}

\chapter*{Барьеры в коммуникации}
\begin{enumerate}
    \item{Недопустимое в рамках кода сообщение: нарушение грамматики языка.}
    \item{Пересечение с другим каналом связи -- помехи.}
    \item{Неоднозначно закодированное сообщение будет неоднозначно раскодировано: <<гуляя по парку, я встретил девушку с собакой, которую я люблю>>.}
    \item{Разные контексты: разговаривают лингвист и философ, первый употребляет слово диск\'{у}рс в специфическом для своей области знания смысле, философ понимает это слово в соответствии со своим бэкграундом.}
    \item{Неверное понимание сообщения адресатом.}
\end{enumerate}
\end{document}

